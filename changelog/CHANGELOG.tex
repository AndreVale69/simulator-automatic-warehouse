\documentclass[a4paper]{article}
\usepackage[italian]{babel}
\usepackage[italian]{isodate}  		% formato delle date in italiano
\usepackage{enumitem}				% gestione delle liste
\usepackage{pifont}					% pacchetto con elenchi carini
\usepackage[x11names]{xcolor}		% colori aggiuntivi
% Link ipertestuali per l'indice
%\usepackage{xcolor}
\usepackage[linkcolor=black, citecolor=blue, urlcolor=cyan]{hyperref}
\hypersetup{
	colorlinks=true
}

\newcommand{\dquotes}[1]{``#1''}

\begin{document}
	\title{Changelog}
	\date{\today}
	\maketitle
	
	\newpage
	
	\section*{Main Improvements}
	\begin{itemize}[label=\ding{51}]
		\item Classe \textsf{Material}:
		\begin{itemize}
			\item Funzione \textsf{gen\_rand\_material} (generazione casuale di materiali), adesso il \emph{barcode} viene generato come esadecimale.
		\end{itemize}
	
		\item Classi \textsf{InsertMaterial} e \textsf{RemoveMaterial}:
		\begin{itemize}
			\item Funzione \textsf{simulate\_action} (simulazione delle azioni), tentativo di ottimizzazione inserendo il codice identico all'interno della superclasse.
		\end{itemize}
	
		\item Classe \textsf{Warehouse} e \textsf{Simulation}:
		\begin{itemize}
			\item Modifica della generazione di eventi, adesso è molto più generico.
			\item Cambio della logica di risorse condivise, adesso vengono utilizzate le \dquotes{risorse condivise} e non più le \emph{pipeline} per comunicare. Inoltre il processo \textsf{Buffer} viene eseguito solamente quando necessario e non rimane sempre in ascolto.
			\item Introdotta la coordinata $y$ all'interno della classe \textsf{Warehouse} e modificate le varie funzioni che ottenevano la coordinata $y$ in modo \emph{hard code}.
		\end{itemize}
	\end{itemize}
	
	\section*{Domande}
	\begin{itemize}
		\item Si blocca tutto nel momento in cui si mettono tutti gli eventi a 1 e nessun drawer in baia.
	\end{itemize}
\end{document}