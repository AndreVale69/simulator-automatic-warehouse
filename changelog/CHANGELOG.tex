\documentclass[a4paper]{article}
\usepackage[italian]{babel}
\usepackage[italian]{isodate}  		% formato delle date in italiano
\usepackage{enumitem}				% gestione delle liste
\usepackage{pifont}					% pacchetto con elenchi carini
\usepackage[x11names]{xcolor}		% colori aggiuntivi
% Link ipertestuali per l'indice
%\usepackage{xcolor}
\usepackage[linkcolor=black, citecolor=blue, urlcolor=cyan]{hyperref}
\hypersetup{
	colorlinks=true
}

\newcommand{\dquotes}[1]{``#1''}

\begin{document}
	\title{Changelog}
	\date{\today}
	\maketitle
	
	\newpage
	
	\section*{Main Improvements}
	\begin{itemize}[label=\ding{51}]
		\item Generato \textsf{UUID} nella classe \textsf{material.py} per il \emph{barcode}.
		
		\item La generazione di elementi casuali avviene selezionando una colonna casuale e inserendo un cassetto con dei materiali oppure vuoto.
		
		\item Il tempo viene calcolato dal basso e non dall'alto di un cassetto.
		
		\item Per la gestione dell'ultima posizione nel magazzino, ho semplicemente aggiunto un parametro per specificare l'altezza di tale cassetto (nel \textsf{JSON}). Poi, ho modificato l'algoritmo che consente l'aggiunta del cassetto, quindi la struttura dati non è stata modificata.
		
		\item Rivoluzionata l'intera simulazione (+ semaforo per gestire l'accesso al carosello, + eventi di rimozione di un materiale), si veda la cartella.
	\end{itemize}
	
	\section*{Domande}
	\begin{itemize}
		\item Si blocca tutto nel momento in cui si mettono tutti gli eventi a 1 e nessun drawer in baia.
	\end{itemize}
\end{document}