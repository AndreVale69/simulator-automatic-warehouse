\documentclass[a4paper]{article}
\usepackage[italian]{babel}
\usepackage[italian]{isodate}  		% formato delle date in italiano
\usepackage{enumitem}				% gestione delle liste
\usepackage{pifont}					% pacchetto con elenchi carini
\usepackage[x11names]{xcolor}		% colori aggiuntivi
% Link ipertestuali per l'indice
%\usepackage{xcolor}
\usepackage[linkcolor=black, citecolor=blue, urlcolor=cyan]{hyperref}
\hypersetup{
	colorlinks=true
}

\newcommand{\dquotes}[1]{``#1''}

\begin{document}
	\title{Changelog}
	\date{\today}
	\maketitle
	
	\newpage
	
	\section*{Main Improvements}
	
	\begin{itemize}[label=\ding{51}]
		\item Generato \textsf{UUID} nella classe \textsf{material.py} per il \emph{barcode}.
		
		\item La generazione di elementi casuali avviene selezionando una colonna casuale e inserendo un cassetto con dei materiali oppure vuoto.
	\end{itemize}
\end{document}