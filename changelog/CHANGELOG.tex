\documentclass[a4paper]{article}
\usepackage[italian]{babel}
\usepackage[italian]{isodate}  		% formato delle date in italiano
\usepackage{enumitem}				% gestione delle liste
\usepackage{pifont}					% pacchetto con elenchi carini
\usepackage[x11names]{xcolor}		% colori aggiuntivi
% Link ipertestuali per l'indice
%\usepackage{xcolor}
\usepackage[linkcolor=black, citecolor=blue, urlcolor=cyan]{hyperref}
\hypersetup{
	colorlinks=true
}

\newcommand{\dquotes}[1]{``#1''}

\begin{document}
	\title{Changelog}
	\date{\today}
	\maketitle
	
	\newpage
	
	\section*{Main Improvements}
	
	\begin{itemize}[label=\ding{51}]
		\item I valori del \textsf{JSON} vengono letti all'interno del costruttore. La funzione legge il file e restituisce il relativo dizionario. Nel costrutto si provvede a caricare tutti i valori nei specifici campi.
		
		\item Rimossi tutti i \textsf{super()} dove era possibile (nei costrutti \textsf{\_\_init\_\_} era impossibile toglierlo).
		
		\item Modificate alcuni nomi delle variabili per essere conformi ai nomi nel \textsf{JSON}.
		
		\item All'interno del costrutto della classe \textsf{Warehouse} avviene la costruzione delle colonne e del carosello passando ai costrutti le varie informazioni.
		
		\item Rimosso il campo del numero delle entries, adesso ogni figlio crea il suo array di oggetti senza utilizzare la superclasse.
		
		\item \textsf{Deepcopy} implementata, ma perché era necessario copiare di nuovo i valori appena passati all'oggetto? Prendi come esempio la classe \textsf{Material} che è semplice.
		
		\item Ogni \textsf{Drawer} ha un campo nella classe che lo collega con la prima \emph{entry} all'interno del \emph{container} delle \emph{entries}. Quindi è stata modificata l'aggiunta dei cassetti in \textsf{Column} e \textsf{Carousel}, è stata migliorata la rimozione dei cassetti e rimossa la funzione di ricerca di un cassetto.
		
		\item Il caricamento del \textsf{buffer} avviene solamente dopo che il ripiano che prende i cassetti ha finito il suo movimento, ovvero nel momento in cui è al centro.
		
		\item Adesso la classe \textsf{Simulation} avvia la simulazione mandando in esecuzione un evento dopo l'altro. Quindi, sono state create $n$-classi per $n$-eventi (o azioni).
		
		\item Le coordinate sono state migliorate. Quindi, la $y$ rappresenta l'altezza (zero dall'alto), come prima, e la $x$ rappresenta l'offset. Più è piccolo il valore di offset, più la colonna è vicina all'uscita e di conseguenza la colonna con offset più piccolo è quella d'uscita.
	\end{itemize}
\end{document}