\documentclass[a4paper]{article}
\usepackage[italian]{babel}
\usepackage[italian]{isodate}  		% formato delle date in italiano
\usepackage{enumitem}				% gestione delle liste
\usepackage{pifont}					% pacchetto con elenchi carini
\usepackage[x11names]{xcolor}		% colori aggiuntivi
% Link ipertestuali per l'indice
%\usepackage{xcolor}
\usepackage[linkcolor=black, citecolor=blue, urlcolor=cyan]{hyperref}
\hypersetup{
	colorlinks=true
}

\newcommand{\dquotes}[1]{``#1''}

\begin{document}
	\title{Changelog}
	\date{\today}
	\maketitle
	
	\newpage
	
	\section*{Main Improvements}
	
	\begin{itemize}[label=\ding{51}]
		\item I valori del \textsf{JSON} vengono letti all'interno del costruttore. La funzione legge il file e restituisce il relativo dizionario. Nel costrutto si provvede a caricare tutti i valori nei specifici campi.
		
		\item Rimossi tutti i \textsf{super()} dove era possibile (nei costrutti \textsf{\_\_init\_\_} era impossibile toglierlo).
		
		\item All'interno di \textsf{drawerContainer.py}:
			\begin{itemize}
				\item Modificato il parametro della funzione add\_drawer da \textsf{DrawerEntry} a \textsf{Drawer}.
				
				\item Modificata la funzione di rimozione, adesso rimuove gli oggetti in modo corretto. Inoltre, nella funzione sono stati aggiunti dei commenti per comprendere meglio il funzionamento.
				
				\item Adesso la funzione crea due liste diverse a seconda dell'oggetto (\textsf{Column} o \textsf{Carousel}).
			\end{itemize}
		
		\item All'interno di \textsf{drawer.py}:
			\begin{itemize}
				\item Risolto il problema dell'inizializzazione senza oggetti \textsf{Material}, modificando l'assegnazione della variabile \textsf{list}.
				
				\item Ottimizzato il ritorno degli oggetti della classe grazie alla risoluzione del problema dell'inizializzazione senza oggetti.
				
				\item Eseguito \dquotes{\emph{override}} dei metodi \textsf{equals (\_\_eq\_\_)} e \textsf{hash (\_\_hash\_\_)}.
			\end{itemize}
		
		\item All'interno di \textsf{warehouse.py}:
			\begin{itemize}
				\item Rimossa la dichiarazione delle istanze dalle colonne dal costrutto \textsf{\_\_init\_\_} e aggiunte nel \textsf{main.py} per aumentare la generalizzazione.
				
				\item Aggiunto il metodo per aggiungere colonne (\textsf{add\_container}).
				
				\item Eseguito \dquotes{\emph{override}} del metodo \textsf{\_\_copy\_\_}.
			\end{itemize}
		
		\item All'interno di \textsf{simulation.py}:
			\begin{itemize}
				\item Copiati i valori del \textsf{JSON} nei campi della classe.
				
				\item Iniziata una prima costruzione degli eventi.
			\end{itemize}
		
		\item All'interno di \textsf{material.py}:
			\begin{itemize}
				\item Eseguito \dquotes{\emph{override}} dei metodi \textsf{equals (\_\_eq\_\_)} e \textsf{hash (\_\_hash\_\_)}.
			\end{itemize}
		
		\item All'interno di \textsf{carousel.py}:
			\begin{itemize}
				\item Modificata la funzione d'aggiunta di un cassetto all'interno del \emph{buffer} o del deposito.
			\end{itemize}
	\end{itemize}
\end{document}